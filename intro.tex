\chapter{Préambule}
Les << algorithmes >> sont omniprésents dans notre vie quotidienne : on les appelle << organisation des tâches >> ou << méthode >>. S'ils sont parfois difficiles à mettre en place, ou contraignants. On les applique le plus souvent par soucis d'efficacité.

Mettons-nous un instant à la place d'un caissier, devant rendre 6€ à un client. Il pourrait le faire avec 6 pièces d'1€, ou bien 3 pièces de 2€ : c'est plus efficace, car demande moins de manipulations. Dans les faits, il a appris à appliquer un algorithme, dit << glouton >> : rendre la plus grosse somme possible par manipulation, donc un billet de 5€ et une pièce de 1€.

Comme les mathématiciens démontrent des théorèmes, les informaticiens établissent la correction d'algorithmes. Peut-on prouver que la méthode << gloutonne >> du rendu de monnaie assure un nombre minimal de manipulations ? Avant tout : peut-on prouver qu'elle permet effectivement de rendre exactement la monnaie dans tous les cas ?

Pour notre exemple dans son cas général, la réponse est : non ! Pour s'en convaincre, il suffit d'imaginer ne plus avoir de pièce d'1€ ni de 50 cent dans la caisse. Si on commence par rendre un billet de 5€, on peut encore finir avec 5 pièces de 20 cent, mais il aurait été plus efficace de rendre 3 pièces de 2€. Encore pire : en imaginant n'avoir que des pièces de 5€ et de 2€ pour rendre cette monnaie, l'algorithme ne donne pas de solution du tout.

Dans la suite, et partant d'algorithmes en arithmétique, nous verrons en quoi sont liés l'informaticien et le mathématicien dans leur recherche de preuve et comment aborder l'algorithmique au collège et au lycée.